\documentclass{article}
\usepackage{amsmath}
\usepackage{graphicx}
\usepackage{listings}
\usepackage{xcolor}
\usepackage{amsmath}
\usepackage{amssymb}
\usepackage{hyperref}
\usepackage{graphicx}
\usepackage{tikz}
\usepackage{geometry}
\usepackage{tcolorbox}
\geometry{margin=0.6in} % margins (default is ~1.5-1.75in)
\usetikzlibrary{positioning}

\usepackage{xcolor} % for dark mdoe
% \definecolor{dark}{HTML}{1F1F1F}
% \pagecolor{dark}
% \color[rgb]{1,1,1}

\definecolor{navyblue}{rgb}{0.0, 0.0, 0.5}

\title{CSC320: Lecture Notes}
\author{exekis}
\date{May 2025} 

% custom colors for code elements
\definecolor{codekeyword}{rgb}{0.0, 0.0, 0.6}
\definecolor{codecomment}{rgb}{0.0, 0.5, 0.0}
\definecolor{codestring}{rgb}{0.6, 0.0, 0.0}

% the listings package
\lstset{
  language=C,                      % language C  
  basicstyle=\ttfamily\small,      % font type and size
  keywordstyle=\color{codekeyword}\bfseries, % keyword color and style
  commentstyle=\color{codecomment}\itshape,  % comment color and style
  stringstyle=\color{codestring},  % string literal color
  numbers=left,                    % line numbers on the left
  numberstyle=\tiny\color{gray},   % line number style
  frame=single,                    % a frame around the code
  breaklines=true,                 % line breaking
  backgroundcolor=\color{white},   % background color
  tabsize=4,                       % tab size
  showspaces=false,                
  showstringspaces=false           
}

\begin{document}

\maketitle

\section*{Week 1}

\subsection*{Types of Machine Learning}
\begin{itemize}
    \item \textbf{Supervised Learning}: Have labeled examples of the correct output/behaviour
    \item \textbf{Unsupervised Learning}: no labeled examples – instead, looking
    for “interesting” patterns in the data
    \item \textbf{Reinforcement Learning}: (not covered) learning system (agent)
      interacts with the world and learns to maximize a scalar reward
      signal
\end{itemize}

\subsection*{Implementing Machine Learning Models and Systems}
\begin{itemize}
    \item Step 1: Understand the problem (is it prediction, learning a good representation).
    \item Step 2: Formulate the problem mathematically (create notation for your inputs and outcomes and model).
    \item Step 3: Formulate an objective function that represents success for your model.
    \item Step 4: Find a strategy to solve the optimization problem on pencil and paper.
    \item Step 5: Translate the algorithm into code.
    \item Step 6: Analyze, iterate, improve design choices in your model and algorithm
\end{itemize}

\subsection*{Nearest Neighbor Methods}
\subsubsection*{Supervised Learning}

% Supervised Learning Process Summary
\begin{tcolorbox}[colback=gray!10!white,colframe=navyblue!70!black,title=Supervised Learning: Step-by-Step]

\textbf{Step 1: Define the Problem}
Build a system that uses training data to make predictions on new inputs.

\centering
\begin{tabular}{|l|l|l|}
\hline
\textbf{Task} & \textbf{Inputs} & \textbf{Labels} \\
\hline
object recognition & image & object category \\
image captioning & image & caption \\
document classification & text document & category \\
speech-to-text & audio waveform & text \\
\hline
\end{tabular}
\par\smallskip
\textit{Examples of supervised learning tasks}
\par
\raggedright

\textbf{Step 2: Mathematical Notation}
\begin{itemize}
  \item Represent inputs as vectors $x^{(i)} \in \mathbb{R}^d$
  \item Training set $D = \{(x^{(1)}, y^{(1)}), \ldots, (x^{(N)}, y^{(N)})\}$
  \item For regression: $y^{(i)} \in \mathbb{R}$
  \item For classification: $y^{(i)} \in \{1,\ldots,C\}$
\end{itemize}

\textbf{Step 3: Optimization Problem}
\begin{itemize}
  \item Given a novel input $x$, find optimal prediction
  \item First approach: Nearest Neighbor algorithm
\end{itemize}

\end{tcolorbox}

\subsubsection*{Nearest Neighbor}


\end{document}